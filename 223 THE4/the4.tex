\documentclass[11pt]{article}
\usepackage[utf8]{inputenc}
\usepackage[dvips]{graphicx}
\usepackage{fancybox}
\usepackage{verbatim}
\usepackage{array}
\usepackage{latexsym}
\usepackage{alltt}
\usepackage{hyperref}
\usepackage{textcomp}
\usepackage{color}
\usepackage{amsmath}
\usepackage{amsfonts}
\usepackage{tikz}
\usepackage{fitch}  % to use fitch
\usepackage{float}
\usepackage[hmargin=3cm,vmargin=5.0cm]{geometry}
%\topmargin=0cm
\topmargin=-2cm
\addtolength{\textheight}{6.5cm}
\addtolength{\textwidth}{2.0cm}
%\setlength{\leftmargin}{-5cm}
\setlength{\oddsidemargin}{0.0cm}
\setlength{\evensidemargin}{0.0cm}
%misc libraries goes here


\begin{document}
\section*{Student Information } 
%Write your full name and id number between the colon and newline
%Put one empty space character after colon and before newline
Full Name : Muhammet Emin Cihangeri \\
Id Number : 2448215 \\


% ND example using fitch
% delete or comment if you intend to use this to generate the vector pdf 

% Write your answers below the section tags
\section*{Answer 1}

	\begin{equation} 
	\begin{split}
		G(x) &= \sum_{n=0}^{\infty} a_{n} \cdot x^{n}  \\
		G(x) - 1 &= \sum_{n=1}^{\infty} a_{n} \cdot x^{n}  \\
		&= \sum_{n=1}^{\infty} (a_{n-1} + 2^{n}) \cdot x^{n}  \\
		&= x \cdot \sum_{n=1}^{\infty} a_{n-1} \cdot x^{n-1} + 2x \cdot \sum_{n=1}^{\infty} 2^{n-1} \cdot x^{n-1}  \\
		&= x \cdot \sum_{n=0}^{\infty} a_{n} \cdot x^{n} + 2x \cdot \sum_{n=0}^{\infty} 2^{n} \cdot x^{n} \\
		G(x) - 1 &= x \cdot G(x) + \frac{2x}{1-2x} \\
		G(x) &= \frac{2}{1-2x} - \frac{1}{1-x} \\
		&= \sum_{n=0}^{\infty} 2^{n+1} \cdot x^{n} - \sum_{n=0}^{\infty} x^{n} \\
		&= \sum_{n=0}^{\infty} (2^{n+1} - 1) \cdot x^{n} \\
		\\
		a_n &= 2^{n+1} - 1 \\
	\end{split}
	\end{equation}

\section*{Answer 2}

\subsection*{a)}
	\setlength{\unitlength}{0.8cm}
	\begin{picture}(16,8)
	\thicklines
	\put(8,3){\circle*{0.1}}
	\put(7.6,3){$3$}
	\put(8,1){\circle*{0.1}}
	\put(7.6,1){$1$}
	\put(10,7){\circle*{0.1}}
	\put(10.3,7){$18$}
	\put(10,3){\circle*{0.1}}
	\put(10.3,3){$2$}
	\put(8,5){\circle*{0.1}}
	\put(7.6,5){$9$}
	\put(8,3){\line(0,-1){2}}
	\put(10,3){\line(0,1){4}}
	\put(8,3){\line(0,1){2}}
	\put(8,5){\line(1,1){2}}
	\put(8,1){\line(1,1){2}}
	\end{picture}

\subsection*{b)}

	$ \begin{bmatrix}
	1 & 1 & 1 & 1 & 1 \\
	0 & 1 & 0 & 0 & 1 \\
	0 & 0 & 1 & 1 & 1 \\
	0 & 0 & 0 & 1 & 1 \\
	0 & 0 & 0 & 0 & 1 \\
	\end{bmatrix} $

\subsection*{c)}

	Yes. Because every pair in the Hasse diagram has an upperbound and a lowerbound.

\subsection*{d)}
	
	The symmetric closure of $R$ is $ R \cup R^{-1} $. This means $ aRb \rightarrow bRa $ \\
	$ \begin{bmatrix}
	1 & 1 & 1 & 1 & 1 \\
	1 & 1 & 0 & 0 & 1 \\
	1 & 0 & 1 & 1 & 1 \\
	1 & 0 & 1 & 1 & 1 \\
	1 & 1 & 1 & 1 & 1 \\
	\end{bmatrix} $

\subsection*{e)}
	
	2 and 9 are not comparable since $ 2 \mid 9 $ is wrong. \\
	3 and 18 are comparable since $ 3 \mid 18 $. \\
 
\section*{Answer 3}

\subsection*{a)}
	By the matrix representation \\
	\\
	$ \begin{bmatrix}
	1 & 0 & 1 & . & 1 \\
	1 & 1 & 1 & . & 0 \\
	0 & 0 & 1 & . & 1 \\
	. & . & . & . & 1 \\
	0 & 1 & 0 & 0 & 1 \\
	\end{bmatrix} $
	\\
	All the diagonals and the upper right elements can either be 1 or 0. Then, the lower left elements will be the opposite of the mirror element of them. Meaning: \\
	$ \forall i, j (a_{ij} = 1 \rightarrow a_{ji} = 0) \wedge (a_{ij} = 0 \rightarrow a_{ji} = 1) $ \\
	Therefore there are $ 2^{\frac{n^2 + n}{2}} $ anti-symmetric binary relations. \\

\subsection*{b)}
	Here, the only difference from 3a is that all the diagonal elements should be 1. Therefore, there are $ 2^{\frac{n^2 - n}{2}} $ reflexive and anti-symmetric binary relations. \\

\end{document}









