\documentclass[11pt]{article}
\usepackage[utf8]{inputenc}
\usepackage[dvips]{graphicx}
\usepackage{fancybox}
\usepackage{verbatim}
\usepackage{array}
\usepackage{latexsym}
\usepackage{alltt}
\usepackage{hyperref}
\usepackage{textcomp}
\usepackage{color}
\usepackage{amsmath}
\usepackage{amsfonts}
\usepackage{tikz}
\usepackage{fitch}  % to use fitch
\usepackage{float}
\usepackage[hmargin=3cm,vmargin=5.0cm]{geometry}
%\topmargin=0cm
\topmargin=-2cm
\addtolength{\textheight}{6.5cm}
\addtolength{\textwidth}{2.0cm}
%\setlength{\leftmargin}{-5cm}
\setlength{\oddsidemargin}{0.0cm}
\setlength{\evensidemargin}{0.0cm}


\begin{document}
\section*{Student Information } 
%Write your full name and id number between the colon and newline
%Put one empty space character after colon and before newline
Full Name : Muhammet Emin Cihangeri \\
Id Number : 2448215 \\


% ND example using fitch
% delete or comment if you intend to use this to generate the vector pdf 

% Write your answers below the section tags
\section*{Q. 1}

\begin{itemize}
            \item Assume that there is a positive integer less then 1. ($ k < 1$)
	   \item Using the theorem given as $ k \in \mathbb{Z}^{+}  \wedge n \in \mathbb{Z}^{+} \rightarrow k^{n}  \in  \mathbb{Z}^{+} $.
	   \item $ k \in \mathbb{Z}^{+}  \wedge 3 \in \mathbb{Z}^{+} \rightarrow k^{3}  \in  \mathbb{Z}^{+} $
	   \item Since $k$ is the least one in the set and $ k^{3} < k $ this makes a contradiction.
	   \item 1 is the least element.
        \end{itemize}{}

\section*{Q.2}
	$ S_{(1,1)} \Rightarrow f_{(1,1)} = 1 $ is the base case. \\
	$ S_{(m,1)}  \Rightarrow x_1 + x_2 \cdots x_m = 1 $   assume  $ f_{(m,1)} = m $ \\
	$ S_{(m+1,1)} \Rightarrow x_1 + x_2 \cdots x_m + x_{m+1} = 1 $ \\
	for  $ x_{m+1} = 0 $ there are m, for $ x_{m+1} = 1 $ there is 1 (rest is all zeros) solutions, $m + 1$ in total.\\
	So, $ f_{(m,1)} = m $  is approved. \\
	\\
	$ S_{(1,n)} \Rightarrow x_1 = n $ assume $ f_{(1,n)}  = 1 $ \\
	$ S_{(1,n+1)} \Rightarrow x_1 = n + 1  $ \\
	Only one solution which is $ x_1 = n + 1 $ \\
	So, $ f_{(1,n)}  = 1 $ is approved. \\
	\\
	In the last step we can say, if our assumption formula is correct for $ S_{(m+1,n)} $ and $ S_{(m,n+1)} $ \\
	and show that $S_{(m+1,n+1)}$ also complies, we can deduce that $S_{(m,n)}$ is correct. \\
	In the case of $ S_{(m+1,n+1)} $, $x_{m+1}$ will be added as a term to $ S_{(m,n+1)} $. For $x_{m+1} = 0$ \\
	the rest will be of number $ f_{(m,n+1)} $. For $x_{m+1} = 1$ the rest will be of number $ f_{(m,n)} $. \\
	It goes on like this and : \\
	$ f_{(m+1,n+1)} = f_{(m,n+1)} + \cdots f_{(m,1)}$ \\
	Then substituting the ones after the first element, it can be written as: \\
	$ f_ {(m+1,n+1)} = f_{(m,n+1)} + f_{(m+1,n)} $ \\
	\\
	$ f_ {(m+1,n+1)} = \frac{(n+m)!}{(n+1)! \cdot (m-1)!} + \frac{(n+m)!}{n! \cdot m!} $ \\
	\\
	$ f_ {(m+1,n+1)} = \frac{(n+m)!}{n! \cdot (m-1)!} \cdot ( \frac{1}{m} + \frac{1}{n+1})$ \\
	\\
	$ f_ {(m+1,n+1)} = \frac{(n+m+1)!}{(n+1)! \cdot (m)!} $ \\
	\\
	As a conclusion, induction worked and the formula is proven. \\

\pagebreak 

\section*{Q.3}
\subsection*{a.)}

	$ \bullet $ --\\
	$ \bullet $   $ \bullet $  \\
	With this orientation, there are 28 possible points to choose from the table as the top point. \\
	-- $ \bullet $ \\
	$ \bullet $  $ \bullet $ \\
	With this orientation, there are 21 possible points to choose from the table as the top point. \\
	$ \bullet $  $ \bullet $ \\
	$ \bullet $ -- \\
	With this orientation, there are 21 possible points to choose from the table as the bottom point. \\
	$ \bullet $  $ \bullet $ \\
	-- $ \bullet $ \\
	With this orientation, there are 21 possible points to choose from the table as the bottom point. \\
	In total there are 91 different choice. \\


\subsection*{b.)}

	There are two possible distributions of 6 elements among four elements. \\
	3 1 1 1   and   2 2 1 1. \\
	For 3 1 1 1 there are 4 orderings and $ C(6, 3) \cdot C(3, 1) \cdot C(2, 1) \cdot C(1, 1) $ combinations. \\
	For 2 2 1 1 there are 6 orderings and $ C(6, 2) \cdot C(4, 2) \cdot C(2, 1) \cdot C(1, 1) $ combinations. \\
	Sum them $ \rightarrow 4 \cdot C(6, 3) \cdot C(3, 1) \cdot C(2, 1) \cdot C(1, 1) + 6 \cdot C(6, 2) \cdot C(4, 2) \cdot C(2, 1) \cdot C(1, 1)$ \\
	1560 in total.  \\
 
\pagebreak

\section*{Q.4}

\subsection*{a.)}

	Take a string of length n. Its first element can be of three types. The following string can be of two types. \\
	First, string of length $ n - 1 $ that starts with a different digit than the first one picked and satisfies the \\
	condition (containing two consecutive same digits). It will be of amount $\frac{2}{3} a_{n-1} $. \\
	Second, a string of length $ n - 1 $ that starts with the same digit as the first one picked and continues with \\
	all kinds of digits. There will be $ 3^{n-2} $ such string. \\
			$ a_n  = 3 . ( \frac{2}{3} a_{n-1} + 3^{n-2}) $ \\
			$ a_n  = 2 a_{n-1} + 3^{n-1} $ \\
\subsection*{b.)}

	$ a_1 = 0 , a_2 = 3 $ \\

\subsection*{c.)}

	$ r - 2 = 0 $ (characteristic equation) \\
	$ r = 2  \Rightarrow  a_h = c_1 2^n$ \\
	$ a_p = c_2 3^n $ (insert $a_p$ to find $c_2$) \\
	$ c_2 3^n = 2 \cdot 3^{n-1} + 3^{n-1} $ \\
	$ c_2 = 1 $ \\
	$ a_p = 3^n $ \\
	Solving for the initial conditions: \\
	$ a_n = c_1 2^n + 3^n $ \\
	$ a_1 = 2 c_1 + 3 = 0$ \\
	$ c_1 = \frac{-3}{2} $ \\
	$ a_n = -3 \cdot 2^{n-1} + 3^{n} $ \\



% add / remove sections etc as needed 
% use your own format

\end{document}

